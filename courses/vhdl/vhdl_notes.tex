% vhdl_notes.tex
\documentclass[twocolumn,60pt]{article}

% Packages for images, URL, indexing, etc.
\usepackage{graphicx} % For including images
\usepackage{hyperref} % For URL and hyperlink
\usepackage{makeidx}  % For creating an index
\usepackage{lipsum}   % For generating dummy text
\makeindex

\begin{document}

\title{VHDL Notes}
\author{Pradeep Venkatachalam}
\date{\today}

\maketitle

\section{Introduction}
A concise guide through the essential principles and concepts of Digital Design, Computer Architecture and RTL Design Practises.

\section{Synchronous vs Asynchronous resets}
\lipsum[1-2] % Generates dummy text, replace with your actual text

\section{Clock Domain Crossing}
\begin{itemize}
    \item \lipsum[6]
    \item \lipsum[7]
\end{itemize}

\section{Computer Arhcitecture}
Inline equations can be added like this $E=mc^2$.
For displayed equations use:
\begin{equation}
    E=mc^2
\end{equation}

\section{Toolchains}
\lipsum[3] % Generates dummy text, replace with your actual text

\subsection{Subsection Example}
\lipsum[4] % Generates dummy text, replace with your actual text

\section{Conclusion}
\lipsum[5] % Generates dummy text, replace with your actual text

\section{Images}
To include images, make sure your image file is in the same directory as your LaTeX document or define the path to it.
% \begin{figure}[h]
% \centering
% \includegraphics[width=0.5\textwidth]{image_name.jpg}
% \caption{Caption for the image.}
% \label{fig:image1}
% \end{figure}

\section{URLs and Hyperlinks}
You can include URLs or hyperlinks like this: \href{http://www.example.com}{Link Text}.
\end{document}



